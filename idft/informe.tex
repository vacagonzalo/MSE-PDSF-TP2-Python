\documentclass[
    11pt,
    spanish,
	a4paper
]{article}
\usepackage[utf8]{inputenc}
\usepackage[spanish]{babel}
\usepackage{graphicx}
\usepackage{authoraftertitle}
\usepackage{float}
\usepackage{caption}
\usepackage{verbatim}
\usepackage{listings}
\captionsetup[table]{labelformat=empty}

\def\doctype{Trabajo práctico}
\title{Transformada de Fourier}
\author{Gonzalo Nahuel Vaca}

\begin{document}

\makeatletter
\begin{titlepage}
	\begin{center}
		\vspace*{1cm}
		
		\Huge
		\textbf{\doctype}
		\vspace{0.5cm}
    
		\LARGE
		\@title
		\vspace{0.5cm}
    
		\textbf{Procesamiento Digital de Señales (fundamentos)}
		
		\vspace{1.5cm}
		
		\textbf{\@author}

		\vspace{1.5cm}

		\includegraphics[width=0.8\textwidth]{img/logoFIUBA.pdf}
		
		\vfill
		Maestría en Sistemas Embebidos\\
		Universidad de Buenos Aires\\
		Argentina\\
		\today
	\end{center}
\end{titlepage}
\makeatother
\newpage

\section{Parte 1}

\begin{lstlisting}[
    basicstyle=\tiny, %or \small or \footnotesize etc.
    ]
import numpy as np
import matplotlib.pyplot as plt
from muestras import muestras as signal


if __name__ == "__main__":
    fs = 200
    N = 100
    M = 100 * N
    td = np.arange(start=0, stop=N / fs, step=1 / fs)

    fig = plt.figure(1)

    sig_sigAxe = fig.add_subplot(2, 2, 1)
    sig_sigAxe.set_xlabel("n")
    sig_sigAxe.set_ylabel("signal(n)")
    sig_sigAxe.grid(True)
    sig_sigAxe.plot(td, signal, "r-")

    # FFT
    sig_fft = np.abs(1 / N * np.fft.fft(signal)) ** 2
    sig_fftAxe = fig.add_subplot(2, 2, 2)
    (sig_fftLn,) = plt.plot([], [], "b-", linewidth=1)
    sig_fftAxe.set_ylim(0, np.max(sig_fft) + 0.05)
    sig_fftAxe.set_xlim(0, fs / 2)
    sig_fftAxe.set_xlabel("k")
    sig_fftAxe.set_ylabel("fft(signal(n))")
    sig_fftAxe.grid(True)
    sig_fftLn.set_data((fs / N) * fs * td, sig_fft)

    td_mod = np.arange(0, (M + N) / fs, 1 / fs)

    zero_padding = np.zeros(M)
    signal_mod = np.concatenate([signal, zero_padding], axis=None)
    sig_mod_sigAxe = fig.add_subplot(2, 2, 3)
    sig_mod_sigAxe.set_xlabel("n")
    sig_mod_sigAxe.set_ylabel("signal_mod(n)")
    sig_mod_sigAxe.grid(True)
    sig_mod_sigAxe.plot(td_mod, signal_mod, "r-")

    # FFT
    sig_mod_fft = np.abs(1 / (N + M) * np.fft.fft(signal_mod)) ** 2
    sig_mod_fftAxe = fig.add_subplot(2, 2, 4)
    (sig_mod_fftLn,) = plt.plot([], [], "b-", linewidth=1)
    sig_mod_fftAxe.set_ylim(0, np.max(sig_mod_fft))
    sig_mod_fftAxe.set_xlim(0, fs / 2)
    sig_mod_fftAxe.set_xlabel("k")
    sig_mod_fftAxe.set_ylabel("fft(signal_mod(n))")
    sig_mod_fftAxe.grid(True)
    sig_mod_fftLn.set_data((fs / (N + M)) * fs * td_mod, sig_mod_fft)

    plt.show()
\end{lstlisting}

En la figura \ref{fig:ifdt} se puede observar el funcionamiento del script.

\begin{figure}[htbp]
	\centering
	\includegraphics[width=\textwidth]{img/ifdt.png}
	\caption{Imagen de las señales.}
	\label{fig:ifdt}
\end{figure}

\end{document}